\documentclass{beamer}
\usepackage{xcolor}

\title{Tail Recursion}
\author{XD}
\date{some random day}

\begin{document}
\frame{\titlepage}

\begin{frame}[fragile]
    \frametitle{Tail Recursion}
    A recursive function is \textbf{tail-recursive}, when the final result of the recursive call is directly the result of the whole recursive function.

    \medskip
    \textbullet \textbf{Not Tail-Recursive}:
    \begin{verbatim}
    sum [] = 0
    sum (x:xs) = x + sum xs
    \end{verbatim}

    \medskip
    \textbullet \textbf{Tail-Recursive}:
    \begin{verbatim}
    sum xs = help 0 xs
        where help acc [] = acc
              help acc (x:xs) = help (acc + x) xs
    \end{verbatim}
\end{frame}

\begin{frame}
    \frametitle{Tail Recursion}
    A tail recursive function {\color{red}{can}} be optimized to avoid building up stack frames. Depenging on the support/option of the compiler, the result may vary.
\end{frame}

\begin{frame}[fragile]
    \frametitle{Stack Frame}
    \begin{verbatim}
    sum [] = 0
    sum (x:xs) = x + sum xs
    \end{verbatim}

    \medskip
    \begin{verbatim}
    sum [5,4,3,2,1]
    = 5 + sum [4,3,2,1]
    = 5 + (4 + sum [3,2,1])
    = 5 + (4 + (3 + sum [2,1]))
    = 5 + (4 + (3 + (2 + sum [1])))
    = 5 + (4 + (3 + (2 + (1 + sum []))))
    = 5 + (4 + (3 + (2 + (1 + 0))))
    = 5 + (4 + (3 + (2 + 1)))
    = ...
    = 15
    \end{verbatim}
\end{frame}

\begin{frame}[fragile]
    \frametitle{Stack Frame}
    When Optimized:
    \begin{verbatim}
    sum xs = help 0 xs
        where help acc [] = acc
              help acc (x:xs) = help (acc + x) xs
    \end{verbatim}

    \medskip
    \begin{verbatim}
    sum [5,4,3,2,1]
    = help 0 [5,4,3,2,1]
    = help 5 [4,3,2,1]
    = help 9 [3,2,1]
    = help 12 [2,1]
    = help 14 [1]
    = help 15 []
    = 15
    \end{verbatim}
\end{frame}

\begin{frame}
    \frametitle{Only When Optimized}
    You \textbf{must} turn on the option of the compiler in order for Tail Recursion being optimized. Otherwise it will behave just like normal functions.

    Of course the compiler/executor itself must also support that. 

    Java's JVM for example does not support this.
\end{frame}

\begin{frame}
    \frametitle{Tail Recursion in Haskell}
    When dealing with lazy evaluated languages, we can't just assume that code will behave the same way as it does in a strict language.

    In Haskell, the function call model is a little different, function calls might not use a new stack frame, so making a function tail-recursive typically isn't as big a deal.
\end{frame}

\begin{frame}[fragile]
    \frametitle{Tail Recursion in C++}

    \textbullet \textbf{Not Tail-Recursive}:
    \begin{verbatim}
    int sum(int n) {
        if (n == 0) return 0;
        return n + sum(n - 1);
    }
    \end{verbatim}

    \medskip
    \textbullet \textbf{Tail-Recursive}:
    \begin{verbatim}
    int sum(int n, int acc = 0) {
        if (n == 0) return acc;
        return sum(n - 1, acc + n);
    }
    \end{verbatim}
\end{frame}

\begin{frame}[fragile]
    \frametitle{Tail Recursion in C++}

    When you turn on the optimization, both will be equivalent to:

    \medskip
    \begin{verbatim}
    int sum(int n) {
        int result = 0;
        for (int i = 0; i <= n; ++i) {
            result += i;
        }
        return result;
    }
    \end{verbatim}

    Yes, essentially they are just for-loops
\end{frame}

\end{document}
