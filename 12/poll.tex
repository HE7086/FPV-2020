\documentclass{beamer}
\usepackage{xcolor}

\title{Dummy Slide For Poll}
\author{XD}
\date{some random day}

\begin{document}
\frame{\titlepage}

\begin{frame}
    \frametitle{Innermost \& Outermost}
    An expression that can be reduced is called {\color{blue}{Redex}} 

    e.g. $(4 + 1) \rightarrow 5$

    \medskip
    \textbullet \textbf{innermost}:
    Redex contains no other redex

    \medskip
    \textbullet \textbf{outermost}:
    Redex is not contained within another redex

\end{frame}

\begin{frame}
    \frametitle{Reduction}
    \textbullet \textbf{primitive operation}

    can only be reduced if it is applied to fully evaluated arguments

    \smallskip
    $1 + (2 + 3)$ as a whole cannot be reduced

    \medskip
    \textbullet \textbf{functions}

    function's reducability depends on its definition

    \smallskip
    $fst (1,2) = 1$

\end{frame}

\begin{frame}
    \frametitle{Redexes}
    1 + {\color{red}{(2 * 3)}}

    \medskip
    innermost \& outermost
\end{frame}

\begin{frame}
    \frametitle{Redexes}
    {\color{red}{1 + (2 * 3)}}

    \medskip
    neither (it self as a whole is not reducible)
\end{frame}

\begin{frame}
    \frametitle{Redexes}
    (1 + 2) + {\color{red}{(2 + 3)}}

    \medskip
    innermost \& outermost
\end{frame}

\begin{frame}
    \frametitle{Redexes}
    {\color{red}{(1 + 2) * (2 + 3)}}

    \medskip
    neither
\end{frame}

\begin{frame}
    \frametitle{Redexes}
    fst (1 + 2, {\color{red}{2 + 3}})

    \medskip
    innermost \& outermost
\end{frame}

\begin{frame}
    \frametitle{Redexes}
    fst ({\color{red}{1 + 2}}, 2 + 3)

    \medskip
    innermost
\end{frame}

\begin{frame}
    \frametitle{Redexes}
    {\color{red}{fst (1 + 2, 2 + 3)}}

    \medskip
    outermost
\end{frame}

\begin{frame}
    \frametitle{Redexes}
    fst (snd (1, {\color{red}{2 + 3}}), 4)

    \medskip
    innermost
\end{frame}

\begin{frame}
    \frametitle{Redexes}
    fst ({\color{red}{snd (1, 2 + 3)}}, 4)

    \medskip
    neither
\end{frame}

\begin{frame}
    \frametitle{Redexes}
    {\color{red}{fst (snd (1, 2 + 3), 4)}}

    \medskip
    outermost
\end{frame}

\begin{frame}
    \frametitle{Redexes}
    (\textbackslash x $\rightarrow$ 1 + x)(2 * 3)

\end{frame}

\begin{frame}
    \frametitle{Redexes}
    (\textbackslash x $\rightarrow$ 1 + x){\color{red}{(2 * 3)}}
    
    \medskip
    innermost
\end{frame}

\begin{frame}
    \frametitle{Redexes}
    {\color{red}{(\textbackslash x $\rightarrow$ 1 + x)(2 * 3)}}

    \medskip
    outermost
\end{frame}

\begin{frame}
    \frametitle{Redexes}
    (\textbackslash x $\rightarrow$ (1 + 2) + x)

    \medskip
    lambda expressions are different.

    In general, one cannot reduce inside it

    even if it's quite obvious, like 1+2 in this case
\end{frame}

\begin{frame}[fragile]
    \frametitle{Why do we care about those?}
    Definition:
    loop = tail loop
    
    \medskip

    \begin{verbatim}
    Innermost reduction:
    fst (1,loop) = fst(1,tail loop)
                 = fst(1,tail(tail loop))
                 = ...
    \end{verbatim}

    \medskip

    \begin{verbatim}
    Outermost reduction:
    fst (1,loop) = 1
    \end{verbatim}

    \medskip
    Outermost reduction terminates as often as possible

    \medskip
    (see lecture slides p346)
\end{frame}

\begin{frame}
    \frametitle{exkurs: Weak Head Normal Form}
    \href{https://wiki.haskell.org/Weak_head_normal_form}{https://wiki.haskell.org/Weak\_head\_normal\_form}

    In order to evaluate an expression, it must be first evaluated to its WHNF.
\end{frame}

\begin{frame}[fragile]
    \frametitle{exkurs: Weak Head Normal Form}
    consider the following expression:

    \begin{verbatim}
    inf :: Eq a => [a]
    inf = inf

    (==) :: Eq a => [a] -> [a] -> Bool
    [] == [] = True
    (x : xs) == (y : ys) = x == y && xs == ys
    _ == _ = False
    \end{verbatim}

    what happens if we call:

    \begin{verbatim}
    inf == inf
    \end{verbatim}

    True? False? not terminating? throw exception?
\end{frame}

\begin{frame}[fragile]
    \frametitle{exkurs: Weak Head Normal Form}
    'inf' will first be evaluated to its WHNF, but none of the case matches([] or (x:xs)), so the definition of inf is applied:

    inf = inf

    \begin{verbatim}
    inf == inf
    (inf <evaluated by definition> inf) == inf
    inf == inf
    ... (not terminating)
    \end{verbatim}

    and it won't terminate

\end{frame}



\end{document}
